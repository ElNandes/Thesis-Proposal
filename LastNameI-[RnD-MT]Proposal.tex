%\documentclass[rnd]{mas_proposal}
 \documentclass[thesis]{mas_proposal}

\usepackage[utf8]{inputenc}
\usepackage{amsmath}
\usepackage{amsfonts}
\usepackage{amssymb}
\usepackage{graphicx}

\title{Exploring Retrieval Augmented Generation Systems For Automatic Short Answer Grading and Assisted Learning}
\author{Elanton Fernandes}
\supervisors{First Supervisor\\Second Supervisor\\Third Supervisor}
\date{April 2024}

% \thirdpartylogo{path/to/your/image}

\begin{document}

\maketitle

\pagestyle{plain}

\section{Introduction}

\subsection{Topic of This Thesis Project}
\begin{itemize}
    \item In order to understand what we wish to do in this thesis we would like to understand the core requirements of what we expect from automated short answer grading and large language models and how a system called as a Retrieval augmented generation comes into the picture.
    \item Any academic institute that conducts examinations would find it beneficial to automate the grading process of exams for each of its students. Now, the underlying problem with this is that, students expect not just a grade but also an explanation of why they received a particular grade for the questions they answered. This is mostly the case when points are deducted from a students answer and they do not receive the full grade.
    \item Large Language Models (LLM) such as GPT-4,  LLama-2, Grok, etc have shown the immense capabilities in the areas of Question Answering, text summarization, Chat, etc. Looking specifically at the task of Question Answers, it is observed that these models can tend to give false information with claims that sound very convincing and can make the user doubt their own knowledge. This phenomena where the Large Language Model generates a response that is factually false but has convincing statements that make the claims sound true is called 'Hallucination' and a model that does that is said to 'Hallucinate'.
    \item Assume the following example question: 'What are different layers in a artificial neural network?'. The correct response to this is: 'A artificial neural network consists of 3 layers, namely, input layer, hidden layers, and output layer'. Lets also make the assumption that in an examination a student gave the following response: 'Neural network comprises of: input layer, convolution layer, and output layer'
    \item Now under ideal circumstances, if we use a Large Language Model and give it the question and student answer, we would expect that the LLM grades this response with two points (Assuming it is a 3 point question) and gives an explanation stating that the student lost one point because the basic model of a neural network does not consist of a convolution layer but a hidden layer.
    \item However, given the possibility that a model can hallucinate, it can also grade the student answer with full point or make a claim that 'the student answer is comletely wrong because a neural network comprises of a network of neurons connected by synapses'
    \item This is where Retrieval Augmented Generation (RAG) will come into the picture. RAG in simple terms is a way in which a user provides documents that it expects the LLM to use when asked a question. So when a question is asked to the LLM, it will retrieve an answer from these documents and answer your question as a complete sentence (and not just copy paste of the reference document).
    \item So in this project we wish to explore the different RAG based systems, such that we can do the task of automated short answer grading for not fully correct answers in the following manner: "The answer you have provided loses X points because the 'XYZ' statement made by you is wrong and the correct answer is in the reference document 'ABC'. "
\end{itemize}

\subsection{Relevance of This Thesis Project}
\begin{itemize}
    \item The results of this thesis will benefit any institute by helping them in grading student examination. This ultimately will save them time and resources that can be used somewhere else effectively.
    \item The results can also benefit any organization that wishes to use a RAG based system to perform QA task on proprietary/confidential documents. This again has the benefit of time and resources saved.
    \item A RAG based system also allows one to not have to fine tune a model on that data, thus saving compute resources and cut down on cost.
 \end{itemize}

\section{Related Work}

\subsection{Survey of Related Work}
\begin{itemize}
    \item What have other people done to solve the problem?
    \item You should reference and briefly discuss at least the ``top twelve'' related works
\end{itemize}

\subsection{Limitation and Deficits in the State of the Art}
\begin{itemize}
    \item List the deficits that you have discovered in the related work and explain them such that a person who is not deep into the technical details can still understand them.
    For each deficit, provide at least two references
    \item You should reference and briefly discuss at least the ``top twelve'' related works
\end{itemize}

\section{Problem Statement}
\begin{itemize}
    \item Which of the deficits are you going to solve?
    \item What is your intended approach?
    \item How will you compare you approach with existing approaches?
\end{itemize}

\section{Project Plan}

\subsection{Work Packages}
\emph{Planning is the replacement of randomness by error.} (Einstein). Very much like you would never start a longer journey without a detailed travel plan, you should not start a project without a carefully though out work plan. A work package is a logical decomposition of a larger piece of work into smaller parts following a ``divide and conquer" strategy. It is very specific to the problem that you are going to address. Refrain from a rather generic decomposition. If your work plan looks similar to those of your school mates, which may address completely different problems then you have not thought carefully enough about how you approach the problem. It is ok to have two generic work packages \emph{Literature Study} and \emph{Project Report}. Discuss your work packages in the ASW seminar.

The bare minimum will include the following packages:
\begin{enumerate}
    \item[WP1] Literature Study
    \item[WP2] ...
    \item[WP3] ...
    \item  ...
    \item[WPy] Evaluation of approach and comparison with similar approaches
    \item[WPz] Project Report
\end{enumerate}

\subsection{Milestones}
Milestones mark the completion of a certain activity or at least a major achievement in an activity. Milestones are also decision points, where you reflect on what you have achieved and what options you have for continuing your work in case you have not achieved what was planned. Above all, milestones have to be measurable. As above, if your milestones are the same as those of your school mates, then you may not have thought carefully enough about how your project shall progress.
\begin{enumerate}
    \item[M1] Literature review completed and best practice identified
    \item[M2] ...
    \item[M3] ...
    \item[M4] Report submission
\end{enumerate}

\subsection{Project Schedule}
Include a Gantt chart here. It doesn't have to be detailed, but it should include the milestones you mentioned above.
Make sure to include the writing of your report throughout the whole project, not just at the end.

\begin{figure}[h!]
    \includegraphics[width=\textwidth]{images/rnd_deliverable_timeline}
    \caption{My figure caption}
    \label{fig:myfigure}
\end{figure}

\subsection{Deliverables}

\subsubsection*{Minimum Viable}
\begin{itemize}
    \item Project results required to get a satisfying or sufficient grade.
\end{itemize}

\subsubsection*{Expected}
\begin{itemize}
    \item Project results required to get a good grade.
\end{itemize}

\subsubsection*{Desired}
\begin{itemize}
    \item Project results required to get an excellent grade.
\end{itemize}

Please note that the final grade will not only depend on the results obtained in your work, but also on how you present the results.

\nocite{*}

\bibliographystyle{plainnat} % Use the plainnat bibliography style
\bibliography{bibliography.bib} % Use the bibliography.bib file as the source of references

\end{document}
